\documentclass[conference]{IEEEtran}
\usepackage[utf8]{inputenc}
\usepackage{amsmath}
\usepackage[pdftex]{graphicx}
\usepackage[numbers]{natbib}
%\usepackage{algorithmic}
%\usepackage{array}
%\usepackage[caption=false,font=normalsize,labelfont=sf,textfont=sf]{subfig}
% *** Do not adjust lengths that control margins, column widths, etc. ***
% *** Do not use packages that alter fonts (such as pslatex).         ***
% There should be no need to do such things with IEEEtran.cls V1.6 and later.
% (Unless specifically asked to do so by the journal or conference you plan
% to submit to, of course. )

% correct bad hyphenation here
\hyphenation{op-tical net-works semi-conduc-tor}

\begin{document}
\title{LELEC2885 : HDR}


% author names and affiliations
% use a multiple column layout for up to three different
% affiliations
\author{
\IEEEauthorblockN{Guillaume Derval}
\IEEEauthorblockA{INFO22MS/G - 68911100}
\and
\IEEEauthorblockN{Xavier Dollé}
\IEEEauthorblockA{INFO22MS/G - xxxxxxxx}
\and
\IEEEauthorblockN{Anthony Gégo}
\IEEEauthorblockA{ELEC22MS/G - 28581100}
\and
\IEEEauthorblockN{Kévin Lhoest}
\IEEEauthorblockA{xx - xxxxxxxx}
}

% make the title area
\maketitle

% As a general rule, do not put math, special symbols or citations
% in the abstract
\begin{abstract}
The abstract
\end{abstract}

\section{Introduction}
Intro
\cite{ma2015perceptual} \cite{zhang2012gradient} \cite{khan2006ghost}

\section{Conclusion}
Conclusion
\newpage
\bibliographystyle{IEEEtranN}
\bibliography{main}

% that's all folks
\end{document}


